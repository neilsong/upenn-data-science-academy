% Options for packages loaded elsewhere
\PassOptionsToPackage{unicode}{hyperref}
\PassOptionsToPackage{hyphens}{url}
%
\documentclass[
]{article}
\title{Data Acquisition, Preparation and EDA}
\author{}
\date{\vspace{-2.5em}}

\usepackage{amsmath,amssymb}
\usepackage{lmodern}
\usepackage{iftex}
\ifPDFTeX
  \usepackage[T1]{fontenc}
  \usepackage[utf8]{inputenc}
  \usepackage{textcomp} % provide euro and other symbols
\else % if luatex or xetex
  \usepackage{unicode-math}
  \defaultfontfeatures{Scale=MatchLowercase}
  \defaultfontfeatures[\rmfamily]{Ligatures=TeX,Scale=1}
\fi
% Use upquote if available, for straight quotes in verbatim environments
\IfFileExists{upquote.sty}{\usepackage{upquote}}{}
\IfFileExists{microtype.sty}{% use microtype if available
  \usepackage[]{microtype}
  \UseMicrotypeSet[protrusion]{basicmath} % disable protrusion for tt fonts
}{}
\makeatletter
\@ifundefined{KOMAClassName}{% if non-KOMA class
  \IfFileExists{parskip.sty}{%
    \usepackage{parskip}
  }{% else
    \setlength{\parindent}{0pt}
    \setlength{\parskip}{6pt plus 2pt minus 1pt}}
}{% if KOMA class
  \KOMAoptions{parskip=half}}
\makeatother
\usepackage{xcolor}
\IfFileExists{xurl.sty}{\usepackage{xurl}}{} % add URL line breaks if available
\IfFileExists{bookmark.sty}{\usepackage{bookmark}}{\usepackage{hyperref}}
\hypersetup{
  pdftitle={Data Acquisition, Preparation and EDA},
  hidelinks,
  pdfcreator={LaTeX via pandoc}}
\urlstyle{same} % disable monospaced font for URLs
\usepackage[margin=1in]{geometry}
\usepackage{color}
\usepackage{fancyvrb}
\newcommand{\VerbBar}{|}
\newcommand{\VERB}{\Verb[commandchars=\\\{\}]}
\DefineVerbatimEnvironment{Highlighting}{Verbatim}{commandchars=\\\{\}}
% Add ',fontsize=\small' for more characters per line
\usepackage{framed}
\definecolor{shadecolor}{RGB}{248,248,248}
\newenvironment{Shaded}{\begin{snugshade}}{\end{snugshade}}
\newcommand{\AlertTok}[1]{\textcolor[rgb]{0.94,0.16,0.16}{#1}}
\newcommand{\AnnotationTok}[1]{\textcolor[rgb]{0.56,0.35,0.01}{\textbf{\textit{#1}}}}
\newcommand{\AttributeTok}[1]{\textcolor[rgb]{0.77,0.63,0.00}{#1}}
\newcommand{\BaseNTok}[1]{\textcolor[rgb]{0.00,0.00,0.81}{#1}}
\newcommand{\BuiltInTok}[1]{#1}
\newcommand{\CharTok}[1]{\textcolor[rgb]{0.31,0.60,0.02}{#1}}
\newcommand{\CommentTok}[1]{\textcolor[rgb]{0.56,0.35,0.01}{\textit{#1}}}
\newcommand{\CommentVarTok}[1]{\textcolor[rgb]{0.56,0.35,0.01}{\textbf{\textit{#1}}}}
\newcommand{\ConstantTok}[1]{\textcolor[rgb]{0.00,0.00,0.00}{#1}}
\newcommand{\ControlFlowTok}[1]{\textcolor[rgb]{0.13,0.29,0.53}{\textbf{#1}}}
\newcommand{\DataTypeTok}[1]{\textcolor[rgb]{0.13,0.29,0.53}{#1}}
\newcommand{\DecValTok}[1]{\textcolor[rgb]{0.00,0.00,0.81}{#1}}
\newcommand{\DocumentationTok}[1]{\textcolor[rgb]{0.56,0.35,0.01}{\textbf{\textit{#1}}}}
\newcommand{\ErrorTok}[1]{\textcolor[rgb]{0.64,0.00,0.00}{\textbf{#1}}}
\newcommand{\ExtensionTok}[1]{#1}
\newcommand{\FloatTok}[1]{\textcolor[rgb]{0.00,0.00,0.81}{#1}}
\newcommand{\FunctionTok}[1]{\textcolor[rgb]{0.00,0.00,0.00}{#1}}
\newcommand{\ImportTok}[1]{#1}
\newcommand{\InformationTok}[1]{\textcolor[rgb]{0.56,0.35,0.01}{\textbf{\textit{#1}}}}
\newcommand{\KeywordTok}[1]{\textcolor[rgb]{0.13,0.29,0.53}{\textbf{#1}}}
\newcommand{\NormalTok}[1]{#1}
\newcommand{\OperatorTok}[1]{\textcolor[rgb]{0.81,0.36,0.00}{\textbf{#1}}}
\newcommand{\OtherTok}[1]{\textcolor[rgb]{0.56,0.35,0.01}{#1}}
\newcommand{\PreprocessorTok}[1]{\textcolor[rgb]{0.56,0.35,0.01}{\textit{#1}}}
\newcommand{\RegionMarkerTok}[1]{#1}
\newcommand{\SpecialCharTok}[1]{\textcolor[rgb]{0.00,0.00,0.00}{#1}}
\newcommand{\SpecialStringTok}[1]{\textcolor[rgb]{0.31,0.60,0.02}{#1}}
\newcommand{\StringTok}[1]{\textcolor[rgb]{0.31,0.60,0.02}{#1}}
\newcommand{\VariableTok}[1]{\textcolor[rgb]{0.00,0.00,0.00}{#1}}
\newcommand{\VerbatimStringTok}[1]{\textcolor[rgb]{0.31,0.60,0.02}{#1}}
\newcommand{\WarningTok}[1]{\textcolor[rgb]{0.56,0.35,0.01}{\textbf{\textit{#1}}}}
\usepackage{longtable,booktabs,array}
\usepackage{calc} % for calculating minipage widths
% Correct order of tables after \paragraph or \subparagraph
\usepackage{etoolbox}
\makeatletter
\patchcmd\longtable{\par}{\if@noskipsec\mbox{}\fi\par}{}{}
\makeatother
% Allow footnotes in longtable head/foot
\IfFileExists{footnotehyper.sty}{\usepackage{footnotehyper}}{\usepackage{footnote}}
\makesavenoteenv{longtable}
\usepackage{graphicx}
\makeatletter
\def\maxwidth{\ifdim\Gin@nat@width>\linewidth\linewidth\else\Gin@nat@width\fi}
\def\maxheight{\ifdim\Gin@nat@height>\textheight\textheight\else\Gin@nat@height\fi}
\makeatother
% Scale images if necessary, so that they will not overflow the page
% margins by default, and it is still possible to overwrite the defaults
% using explicit options in \includegraphics[width, height, ...]{}
\setkeys{Gin}{width=\maxwidth,height=\maxheight,keepaspectratio}
% Set default figure placement to htbp
\makeatletter
\def\fps@figure{htbp}
\makeatother
\setlength{\emergencystretch}{3em} % prevent overfull lines
\providecommand{\tightlist}{%
  \setlength{\itemsep}{0pt}\setlength{\parskip}{0pt}}
\setcounter{secnumdepth}{-\maxdimen} % remove section numbering
\AtBeginSection[]{\begin{frame}\tableofcontents[currentsection]\end{frame}}
\AtBeginSubsection[]{\begin{frame}\tableofcontents[currentsubsection]\end{frame}{}}
\ifLuaTeX
  \usepackage{selnolig}  % disable illegal ligatures
\fi

\begin{document}
\maketitle

\hypertarget{introduction-and-objectives}{%
\section{Introduction and
Objectives}\label{introduction-and-objectives}}

\begin{itemize}
\tightlist
\item
  \textbf{Data Science} connects statistics, computer science, and
  domain knowledge.
\item
  We look for patterns \& reasons for differences/changes in datasets.
\end{itemize}

\begin{longtable}[]{@{}
  >{\raggedright\arraybackslash}p{(\columnwidth - 0\tabcolsep) * \real{0.06}}@{}}
\toprule
\endhead
 \\
- Once we have the data, we proceed to extract useful information - BUT
we must understand the data first \\
- In this lecture: - basic data acquisition/preparation - understand the
nature of the data via \textbf{exploratory data analysis (EDA)} -
explore plausible variable relationships - \emph{We defer formal
modeling for later} \\
\bottomrule
\end{longtable}

\frametitle{Introduction and Objectives}

\begin{itemize}
\tightlist
\item
  Data mining tools:

  \begin{itemize}
  \tightlist
  \item
    expanding dramatically in the past 20 yrs
  \end{itemize}
\item
  R

  \begin{itemize}
  \tightlist
  \item
    popular among data scientists \& in academia
  \item
    open-source
  \item
    most SOTA methods have R package implementations
  \end{itemize}
\end{itemize}

\begin{longtable}[]{@{}
  >{\raggedright\arraybackslash}p{(\columnwidth - 0\tabcolsep) * \real{0.06}}@{}}
\toprule
\endhead
 \\
- This module will focus on \textbf{data preparation}, \textbf{data
cleaning}, and \textbf{exploratory data analysis (EDA)}. - R and R
Markdown will be used. See advanced\_R\_tutorial.Rmd for extremely
useful EDA tools such as \texttt{dplyr}, \texttt{ggplot},
\texttt{data.table}, and more. \\
\bottomrule
\end{longtable}

\frametitle{Contents}

\begin{enumerate}
\def\labelenumi{\arabic{enumi}.}
\setcounter{enumi}{-1}
\tightlist
\item
  Suggested extra readings/doing:

  \begin{itemize}
  \tightlist
  \item
    run and study Get\_staRted.Rmd
  \item
    run and study advanced\_R\_tutorial.Rmd and
    advanced\_R\_tutorial.html
  \item
    read \texttt{50\ years\ of\ data\ science} and
    \texttt{Teaching\ data\ science} available in Canvas. (skip this)
  \item
    Data set: \texttt{MLPayData\_Total.csv}
  \end{itemize}
\item
  Case Study: Billion dollar Billy Beane
\item
  Study flow:

  \begin{itemize}
  \tightlist
  \item
    Study design
  \item
    Gathering data
  \item
    Process data (tidy data)
  \item
    Exploratory Data Analysis (EDA)
  \item
    Conclusion/Challenges
  \end{itemize}
\item
  R functions

  \begin{itemize}
  \tightlist
  \item
    basic r functions
  \item
    dplyr
  \item
    ggplot
  \end{itemize}
\end{enumerate}

\begin{longtable}[]{@{}
  >{\raggedright\arraybackslash}p{(\columnwidth - 0\tabcolsep) * \real{0.06}}@{}}
\toprule
\endhead
 \\
\textbf{Background} \\
- In the article
\href{https://fivethirtyeight.com/features/billion-dollar-billy-beane/}{Billion
Dollar Billy Beane}: - author: Benjamin Morris - studies a regression of
performance vs.~total payroll from all 30 teams over a 17 yr period -
Oakland A's performance is comparable to the Boston Red Sox - therefore
argues that Billy Beane is worth \$12 million over 5 years - Article
link:
\url{https://fivethirtyeight.com/features/billion-dollar-billy-beane/} \\
\bottomrule
\end{longtable}

\frametitle{Case Study: Baseball}

\textbf{Objectives}

\begin{enumerate}
\def\labelenumi{\arabic{enumi}.}
\tightlist
\item
  Reproduce Benjamin's study.
\end{enumerate}

\begin{itemize}
\tightlist
\item
  Is Billy Beane (Oakland A's GM) worth 12.5 million dollars for a
  period of 5 years, as argued in the article?
\item
  We challenge Benjamin's reasoning behind his argument.
\end{itemize}

\begin{enumerate}
\def\labelenumi{\arabic{enumi}.}
\setcounter{enumi}{1}
\tightlist
\item
  Explore general questions:
\end{enumerate}

\begin{itemize}
\tightlist
\item
  How does pay and performance relate to each other?
\item
  Will a team perform better when they are paid more?
\end{itemize}

\begin{longtable}[]{@{}
  >{\raggedright\arraybackslash}p{(\columnwidth - 0\tabcolsep) * \real{0.06}}@{}}
\toprule
\endhead
 \\
\textbf{Data:} \texttt{MLPayData\_Total.csv}, consists of winning
records and the payroll of all 30 ML teams from 1998 to 2014 (17 years).
There are 162 games in each season. \\
The variables included are: \\
* \texttt{team\ name}: team names * \texttt{p2014}: total pay in 2014 in
\textbf{millions} and other years indicated by year * \texttt{X2014}:
number of games won in 2014 and other years labeled *
\texttt{X2014.pct}: percent winning in 2014 and other years (We only
need one of the two variables from above.) \\
\#\# Data Preparation \\
\bottomrule
\end{longtable}

\frametitle{Data Preparation}

\begin{alertblock}{}
Before we do any analysis, it is a \textbf{MUST} that we take a look at the data.
\end{alertblock}

In particular, we will try to:

\textbf{Tidy the data:}

\begin{itemize}
\tightlist
\item
  Import data/Data preparation
\item
  Data format
\item
  Missing values/peculiarity
\item
  Understand the variables: unit, format, unusual values, etc.
\item
  Put data into a standard data format
\end{itemize}

Columns: variables\\
Rows: subjects

\begin{longtable}[]{@{}
  >{\raggedright\arraybackslash}p{(\columnwidth - 0\tabcolsep) * \real{0.06}}@{}}
\toprule
\endhead
 \\
\textbf{What is in the dataset?} Take a quick look at the data. Pay
attention to what is in the data, any missing values, and the variable
format. \tiny \\
\texttt{r\ names(datapay)} \\
\texttt{\#\#\ \ {[}1{]}\ "Team.name.2014"\ "p1998"\ \ \ \ \ \ \ \ \ \ "p1999"\ \ \ \ \ \ \ \ \ \ "p2000"\ \ \ \ \ \ \ \ \ \ "p2001"\ \ \ \ \ \ \ \ \ \ "p2002"\ \ \ \ \ \ \ \ \ \ "p2003"\ \ \ \ \ \ \ \ \ \ "p2004"\ \ \ \ \ \ \ \ \ \ "p2005"\ \ \ \ \ \ \ \ \ \ "p2006"\ \ \ \ \ \ \ \ \ \ "p2007"\ \ \ \ \ \ \ \ \ \ "p2008"\ \ \ \ \ \ \ \ \ \ "p2009"\ \ \ \ \ \ \ \ \ \ "p2010"\ \ \ \ \ \ \ \ \ \ "p2011"\ \#\#\ {[}16{]}\ "p2012"\ \ \ \ \ \ \ \ \ \ "p2013"\ \ \ \ \ \ \ \ \ \ "p2014"\ \ \ \ \ \ \ \ \ \ "X2014"\ \ \ \ \ \ \ \ \ \ "X2013"\ \ \ \ \ \ \ \ \ \ "X2012"\ \ \ \ \ \ \ \ \ \ "X2011"\ \ \ \ \ \ \ \ \ \ "X2010"\ \ \ \ \ \ \ \ \ \ "X2009"\ \ \ \ \ \ \ \ \ \ "X2008"\ \ \ \ \ \ \ \ \ \ "X2007"\ \ \ \ \ \ \ \ \ \ "X2006"\ \ \ \ \ \ \ \ \ \ "X2005"\ \ \ \ \ \ \ \ \ \ "X2004"\ \ \ \ \ \ \ \ \ \ "X2003"\ \#\#\ {[}31{]}\ "X2002"\ \ \ \ \ \ \ \ \ \ "X2001"\ \ \ \ \ \ \ \ \ \ "X2000"\ \ \ \ \ \ \ \ \ \ "X1999"\ \ \ \ \ \ \ \ \ \ "X1998"\ \ \ \ \ \ \ \ \ \ "X2014.pct"\ \ \ \ \ \ "X2013.pct"\ \ \ \ \ \ "X2012.pct"\ \ \ \ \ \ "X2011.pct"\ \ \ \ \ \ "X2010.pct"\ \ \ \ \ \ "X2009.pct"\ \ \ \ \ \ "X2008.pct"\ \ \ \ \ \ "X2007.pct"\ \ \ \ \ \ "X2006.pct"\ \ \ \ \ \ "X2005.pct"\ \#\#\ {[}46{]}\ "X2004.pct"\ \ \ \ \ \ "X2003.pct"\ \ \ \ \ \ "X2002.pct"\ \ \ \ \ \ "X2001.pct"\ \ \ \ \ \ "X2000.pct"\ \ \ \ \ \ "X1999.pct"\ \ \ \ \ \ "X1998.pct"}
\normalsize Is anything bothering you? We may want to change names of
teams to a shorter, neater name. \\
\bottomrule
\end{longtable}

\frametitle{Data Preparation}

Everything seems to be OK at the moment other than changing one variable
name.\\
\tiny

\begin{Shaded}
\begin{Highlighting}[]
\CommentTok{\#summary(datapay)}
\FunctionTok{summary}\NormalTok{(datapay)[}\DecValTok{1}\SpecialCharTok{:}\DecValTok{10}\NormalTok{] }\CommentTok{\# quick summary. missing values may be shown}
\end{Highlighting}
\end{Shaded}

\begin{verbatim}
##  [1] "Length:30         " "Class :character  " "Mode  :character  " NA                   NA                   NA                   "Min.   : 8.3  "     "1st Qu.:27.7  "     "Median :43.9  "     "Mean   :41.1  "
\end{verbatim}

\begin{Shaded}
\begin{Highlighting}[]
\FunctionTok{str}\NormalTok{(datapay) }\CommentTok{\# data structure}
\end{Highlighting}
\end{Shaded}

\begin{verbatim}
## 'data.frame':    30 obs. of  52 variables:
##  $ Team.name.2014: chr  "Arizona Diamondbacks" "Atlanta Braves" "Baltimore Orioles" "Boston Red Sox" ...
##  $ p1998         : num  31.6 61.7 71.9 59.5 49.8 ...
##  $ p1999         : num  70.5 74.9 72.2 71.7 42.1 ...
##  $ p2000         : num  81 84.5 81.4 77.9 60.5 ...
##  $ p2001         : num  81.2 91.9 72.4 109.6 64 ...
##  $ p2002         : num  102.8 93.5 60.5 108.4 75.7 ...
##  $ p2003         : num  80.6 106.2 73.9 99.9 79.9 ...
##  $ p2004         : num  70.2 88.5 51.2 125.2 91.1 ...
##  $ p2005         : num  63 85.1 74.6 121.3 87.2 ...
##  $ p2006         : num  59.7 90.2 72.6 120.1 94.4 ...
##  $ p2007         : num  52.1 87.3 93.6 143 99.7 ...
##  $ p2008         : num  66.2 102.4 67.2 133.4 118.3 ...
##  $ p2009         : num  73.6 96.7 67.1 122.7 135.1 ...
##  $ p2010         : num  60.7 84.4 81.6 162.7 146.9 ...
##  $ p2011         : num  53.6 87 85.3 161.4 125.5 ...
##  $ p2012         : num  74.3 83.3 81.4 173.2 88.2 ...
##  $ p2013         : num  89.1 89.8 91 150.7 104.3 ...
##  $ p2014         : num  113 111 107 163 89 ...
##  $ X2014         : int  64 79 96 71 73 73 76 85 66 90 ...
##  $ X2013         : int  81 96 85 97 66 63 90 92 74 93 ...
##  $ X2012         : int  81 94 93 69 61 85 97 68 64 88 ...
##  $ X2011         : int  94 89 69 90 71 79 79 80 73 95 ...
##  $ X2010         : int  65 91 66 89 75 88 91 69 83 81 ...
##  $ X2009         : int  70 86 64 95 83 79 78 65 92 86 ...
##  $ X2008         : int  82 72 68 95 97 89 74 81 74 74 ...
##  $ X2007         : int  90 84 69 96 85 72 72 96 90 88 ...
##  $ X2006         : int  76 79 70 86 66 90 80 78 76 95 ...
##  $ X2005         : int  77 90 74 95 79 99 73 93 67 71 ...
##  $ X2004         : int  51 96 78 98 89 83 76 80 68 72 ...
##  $ X2003         : int  84 101 71 95 88 86 69 68 74 43 ...
##  $ X2002         : int  98 101 67 93 67 81 78 74 73 55 ...
##  $ X2001         : int  92 88 63 82 88 83 66 91 73 66 ...
##  $ X2000         : int  85 95 74 85 65 95 85 90 82 79 ...
##  $ X1999         : int  100 103 78 94 67 75 96 97 72 69 ...
##  $ X1998         : int  65 106 79 92 90 80 77 89 77 65 ...
##  $ X2014.pct     : num  0.395 0.488 0.593 0.438 0.451 ...
##  $ X2013.pct     : num  0.5 0.593 0.525 0.599 0.407 ...
##  $ X2012.pct     : num  0.5 0.58 0.574 0.426 0.377 ...
##  $ X2011.pct     : num  0.58 0.549 0.426 0.556 0.438 ...
##  $ X2010.pct     : num  0.401 0.562 0.407 0.549 0.463 ...
##  $ X2009.pct     : num  0.432 0.531 0.395 0.586 0.516 ...
##  $ X2008.pct     : num  0.506 0.444 0.422 0.586 0.602 ...
##  $ X2007.pct     : num  0.556 0.519 0.426 0.593 0.525 ...
##  $ X2006.pct     : num  0.469 0.488 0.432 0.531 0.407 ...
##  $ X2005.pct     : num  0.475 0.556 0.457 0.586 0.488 ...
##  $ X2004.pct     : num  0.315 0.593 0.481 0.605 0.549 ...
##  $ X2003.pct     : num  0.519 0.623 0.438 0.586 0.543 ...
##  $ X2002.pct     : num  0.605 0.631 0.414 0.574 0.414 ...
##  $ X2001.pct     : num  0.568 0.543 0.391 0.509 0.543 ...
##  $ X2000.pct     : num  0.525 0.586 0.457 0.525 0.401 ...
##  $ X1999.pct     : num  0.617 0.636 0.481 0.58 0.414 ...
##  $ X1998.pct     : num  0.401 0.654 0.488 0.568 0.552 ...
\end{verbatim}

\begin{longtable}[]{@{}
  >{\raggedright\arraybackslash}p{(\columnwidth - 0\tabcolsep) * \real{0.06}}@{}}
\toprule
\endhead
 \\
The original format of the dataset \texttt{MLPayData\_Total.csv} is not
in a desirable format. Each row lists multiple results. Also the
variable \texttt{year} is missing. \tiny \\
\texttt{r\ datapay{[}1:4,\ 1:5{]}\ \ \#\ list\ a\ few\ lines\ (subsetting)} \\
\texttt{\#\#\ \ \ \ \ \ \ \ \ \ \ \ \ \ \ \ \ \ \ team\ p1998\ p1999\ p2000\ p2001\ \#\#\ 1\ Arizona\ Diamondbacks\ \ 31.6\ \ 70.5\ \ 81.0\ \ 81.2\ \#\#\ 2\ \ \ \ \ \ \ Atlanta\ Braves\ \ 61.7\ \ 74.9\ \ 84.5\ \ 91.9\ \#\#\ 3\ \ \ \ Baltimore\ Orioles\ \ 71.9\ \ 72.2\ \ 81.4\ \ 72.4\ \#\#\ 4\ \ \ \ \ \ \ Boston\ Red\ Sox\ \ 59.5\ \ 71.7\ \ 77.9\ 109.6} \\
\texttt{r\ \#datapay\$team\ \#\ get\ variables} \\
\bottomrule
\end{longtable}

\frametitle{Data Preparation: Reshape the data}

We would like to reshape the data into the following table format:

\begin{itemize}
\tightlist
\item
  columns (variables) contain all variables\\
\item
  each row records one result(s)
\end{itemize}

In our case we have four variables: team, year, pay, win\_number and
win\_percentage. Let's rearrange the data into the following form:

team \textbar{} year \textbar{} payroll \textbar{} win\_number
\textbar{} win\_percentage

\begin{longtable}[]{@{}
  >{\raggedright\arraybackslash}p{(\columnwidth - 0\tabcolsep) * \real{0.06}}@{}}
\toprule
\endhead
 \\
Let's create the other variables. \tiny \\
```r win\_num \textless- datapay \%\textgreater\% \# create variable:
win\_num and year select(team, X1998:X2014) \%\textgreater\%
pivot\_longer(cols = X1998:X2014, names\_to = ``year'', names\_prefix =
``X'', values\_to = ``win\_num'') \\
win\_pct \textless- datapay \%\textgreater\% \# create variable:
win\_pct and year select(team, X1998.pct:X2014.pct) \%\textgreater\%
pivot\_longer(cols = X1998.pct:X2014.pct, names\_to = ``year'',
names\_prefix = ``X'', values\_to = ``win\_pct'') \%\textgreater\%
mutate(year = substr(year, 1, 4)) ``` \\
\bottomrule
\end{longtable}

\frametitle{Data Preparation: Reshape the data  }

Finally, we join the tables into team, year, payroll, win\_num, and
win\_pct.\\
\tiny

\begin{Shaded}
\begin{Highlighting}[]
\NormalTok{datapay\_long }\OtherTok{\textless{}{-}}\NormalTok{ payroll }\SpecialCharTok{\%\textgreater{}\%} 
  \FunctionTok{inner\_join}\NormalTok{(win\_num, }\AttributeTok{by =} \FunctionTok{c}\NormalTok{(}\StringTok{"team"}\NormalTok{, }\StringTok{"year"}\NormalTok{)) }\SpecialCharTok{\%\textgreater{}\%}
  \FunctionTok{inner\_join}\NormalTok{(win\_pct, }\AttributeTok{by =} \FunctionTok{c}\NormalTok{(}\StringTok{"team"}\NormalTok{, }\StringTok{"year"}\NormalTok{)) }
\FunctionTok{head}\NormalTok{(datapay\_long, }\DecValTok{2}\NormalTok{)  }\CommentTok{\# see first 2 rows}
\end{Highlighting}
\end{Shaded}

\begin{verbatim}
## # A tibble: 2 x 5
##   team                 year  payroll win_num win_pct
##   <chr>                <chr>   <dbl>   <int>   <dbl>
## 1 Arizona Diamondbacks 1998     31.6      65   0.401
## 2 Arizona Diamondbacks 1999     70.5     100   0.617
\end{verbatim}

\begin{longtable}[]{@{}
  >{\raggedright\arraybackslash}p{(\columnwidth - 0\tabcolsep) * \real{0.06}}@{}}
\toprule
\endhead
 \\
After processing, save this cleaned data file into a new table called
\texttt{baseball.csv}. Let's output this table to the /data folder in
our working folder. From now on we will only use the data file
\texttt{baseball}. \tiny \\
\texttt{r\ write.csv(datapay\_long,\ "data/baseball.csv",\ row.names\ =\ F)}
\normalsize \\
\# Exploratory Data Analysis (EDA) - All the analyses done in this
lecture will be exploratory. - The goal is to
\alert{see what information we might be able to extract} so that it will
support the goal of our study. This is an extremely important first step
of the data analyses. - We try to understand the data, summarize the
data, then finally explore the relationships among the variables through
useful visualization. \\
\#\# Part I: Analyze aggregated variables \\
In this section: \\
- Try to use aggregated information such as: - the total pay for each
team - average performance - Look for the relationship between
performance and the payroll as suggested in Morris's post. \\
\bottomrule
\end{longtable}

\frametitle{Input the data}

First, input the clean data \texttt{baseball} and quickly explore the
data.\\
Everything seems fine: no missing values, names of variables are good.
The class of each variable matches its nature. (numeric, factor,
characters\ldots)\\
\tiny

\begin{Shaded}
\begin{Highlighting}[]
\NormalTok{baseball }\OtherTok{\textless{}{-}} \FunctionTok{read.csv}\NormalTok{(}\StringTok{"data/baseball.csv"}\NormalTok{, }\AttributeTok{header =} \ConstantTok{TRUE}\NormalTok{, }\AttributeTok{stringsAsFactors =}\NormalTok{ F)}
\FunctionTok{names}\NormalTok{(baseball)}
\FunctionTok{str}\NormalTok{(baseball)}
\FunctionTok{summary}\NormalTok{(baseball)}
\CommentTok{\#View(baseball)}
\end{Highlighting}
\end{Shaded}

\begin{verbatim}
## [1] "team"    "year"    "payroll" "win_num" "win_pct"
## 'data.frame':    510 obs. of  5 variables:
##  $ team   : chr  "Arizona Diamondbacks" "Arizona Diamondbacks" "Arizona Diamondbacks" "Arizona Diamondbacks" ...
##  $ year   : int  1998 1999 2000 2001 2002 2003 2004 2005 2006 2007 ...
##  $ payroll: num  31.6 70.5 81 81.2 102.8 ...
##  $ win_num: int  65 100 85 92 98 84 51 77 76 90 ...
##  $ win_pct: num  0.401 0.617 0.525 0.568 0.605 ...
##      team                year         payroll         win_num       win_pct     
##  Length:510         Min.   :1998   Min.   :  8.3   Min.   : 43   Min.   :0.265  
##  Class :character   1st Qu.:2002   1st Qu.: 51.3   1st Qu.: 72   1st Qu.:0.444  
##  Mode  :character   Median :2006   Median : 73.3   Median : 81   Median :0.500  
##                     Mean   :2006   Mean   : 78.1   Mean   : 81   Mean   :0.500  
##                     3rd Qu.:2010   3rd Qu.: 95.0   3rd Qu.: 90   3rd Qu.:0.556  
##                     Max.   :2014   Max.   :235.3   Max.   :116   Max.   :0.716
\end{verbatim}

\begin{longtable}[]{@{}
  >{\raggedright\arraybackslash}p{(\columnwidth - 0\tabcolsep) * \real{0.06}}@{}}
\toprule
\endhead
 \\
To summarize a continuous variable (such as \texttt{payroll\_total} or
\texttt{win\_pct\_ave}), we use the following measurements: \\
+ \textbf{Center}: sample mean/median + \textbf{Spread}: sample standard
deviation + \textbf{Range}: minimum and maximum + \textbf{Distribution}:
quantiles \\
\bottomrule
\end{longtable}

\frametitle{Descriptive statistics}

First, let us take a look at \texttt{payroll\_total}.

\textbf{Base \texttt{R} way:} \tiny

\begin{Shaded}
\begin{Highlighting}[]
\FunctionTok{mean}\NormalTok{(data\_agg}\SpecialCharTok{$}\NormalTok{payroll\_total)}
\FunctionTok{sd}\NormalTok{(data\_agg}\SpecialCharTok{$}\NormalTok{payroll\_total) }
\FunctionTok{quantile}\NormalTok{(data\_agg}\SpecialCharTok{$}\NormalTok{payroll\_total, }\AttributeTok{prob =} \FunctionTok{seq}\NormalTok{(}\DecValTok{0}\NormalTok{, }\DecValTok{1}\NormalTok{, }\FloatTok{0.25}\NormalTok{))}
\FunctionTok{median}\NormalTok{(data\_agg}\SpecialCharTok{$}\NormalTok{payroll\_total)}
\FunctionTok{max}\NormalTok{(data\_agg}\SpecialCharTok{$}\NormalTok{payroll\_total)}
\FunctionTok{min}\NormalTok{(data\_agg}\SpecialCharTok{$}\NormalTok{payroll\_total)}
\FunctionTok{summary}\NormalTok{(data\_agg}\SpecialCharTok{$}\NormalTok{payroll\_total)}
\end{Highlighting}
\end{Shaded}

\begin{verbatim}
## [1] 1.33
## [1] 0.45
##    0%   25%   50%   75%  100% 
## 0.698 1.022 1.264 1.517 2.857 
## [1] 1.26
## [1] 2.86
## [1] 0.698
##    Min. 1st Qu.  Median    Mean 3rd Qu.    Max. 
##   0.698   1.022   1.264   1.328   1.517   2.857
\end{verbatim}

\begin{longtable}[]{@{}
  >{\raggedleft\arraybackslash}p{(\columnwidth - 0\tabcolsep) * \real{0.06}}@{}}
\toprule
\endhead
 \\
Find the team with the max/min payroll. \\
\textbf{Base \texttt{R} way:} \tiny \\
\texttt{r\ data\_agg\$team{[}which.max(data\_agg\$payroll\_total){]}} \\
\texttt{\#\#\ {[}1{]}\ "New\ York\ Yankees"} \\
\texttt{r\ data\_agg\$team{[}which.min(data\_agg\$payroll\_total){]}} \\
\texttt{\#\#\ {[}1{]}\ "Miami\ Marlins"} \\
\bottomrule
\end{longtable}

\frametitle{Descriptive statistics}

Rearrange the data to see the ranks of team by \texttt{payroll}.

But we can easily rearrange the whole data set \texttt{data\_agg} by
ordering one variable, say \texttt{payroll\_total}.

\textbf{Base \texttt{R} way:} \tiny

\begin{Shaded}
\begin{Highlighting}[]
\CommentTok{\#To rank teams by payroll in decreasing order}
\FunctionTok{arrange}\NormalTok{(data\_agg, }\FunctionTok{desc}\NormalTok{(payroll\_total))[}\DecValTok{1}\SpecialCharTok{:}\DecValTok{5}\NormalTok{,] }\CommentTok{\#default decs=T}
\end{Highlighting}
\end{Shaded}

\begin{verbatim}
## # A tibble: 5 x 3
##   team                  payroll_total win_pct_ave
##   <chr>                         <dbl>       <dbl>
## 1 New York Yankees               2.86       0.594
## 2 Boston Red Sox                 2.10       0.553
## 3 Los Angeles Dodgers            1.87       0.529
## 4 New York Mets                  1.72       0.502
## 5 Philadelphia Phillies          1.69       0.519
\end{verbatim}

\begin{Shaded}
\begin{Highlighting}[]
\CommentTok{\#arrange(data\_agg, win\_pct\_ave) \# default???}
\CommentTok{\#arrange(data\_agg, {-}desc(payroll\_total))[1:5,] }
\end{Highlighting}
\end{Shaded}

\begin{longtable}[]{@{}
  >{\raggedright\arraybackslash}p{(\columnwidth - 0\tabcolsep) * \real{0.06}}@{}}
\toprule
\endhead
 \\
\textbf{\texttt{ggplot} plots:} \tiny \\
```r p1 \textless- ggplot(data\_agg) + geom\_histogram(aes(x =
payroll\_total), bins = 10, fill = ``blue'') + labs( title = ``Histogram
of Payroll (frequency)'', x = ``Payroll'' , y = ``Frequency'') \\
p2 \textless- ggplot(data\_agg) + geom\_histogram(aes(x =
payroll\_total, y = ..density..), bins = 10, fill = ``light blue'') +
labs( title = ``Histogram of Payroll (percentage)'', x = ``Payroll'' , y
= ``Percentage'') \\
grid.arrange(p1, p2, ncol = 2) \# facet the two plots side by side
``` \\
\includegraphics{DataPreparationEDASlides_files/figure-latex/facet hist-1.pdf} \\
\normalsize Notice, the two plots above look identical but with
different \texttt{y-scale}. \\
\bottomrule
\end{longtable}

\frametitle{Displaying variables: Boxplots}

A \textbf{boxplot} captures the spread by showing median, quantiles and
outliers:

\textbf{\texttt{ggplot} plots:}\\
\tiny

\begin{Shaded}
\begin{Highlighting}[]
\FunctionTok{ggplot}\NormalTok{(data\_agg) }\SpecialCharTok{+} 
  \FunctionTok{geom\_boxplot}\NormalTok{(}\FunctionTok{aes}\NormalTok{(}\AttributeTok{x=}\StringTok{""}\NormalTok{, }\AttributeTok{y=}\NormalTok{payroll\_total)) }\SpecialCharTok{+} 
  \FunctionTok{labs}\NormalTok{(}\AttributeTok{title=}\StringTok{"Boxplot of Pay Total"}\NormalTok{, }\AttributeTok{x=}\StringTok{""}\NormalTok{) }
\end{Highlighting}
\end{Shaded}

\includegraphics{DataPreparationEDASlides_files/figure-latex/boxplot-1.pdf}

\begin{longtable}[]{@{}
  >{\raggedright\arraybackslash}p{(\columnwidth - 0\tabcolsep) * \real{0.06}}@{}}
\toprule
\endhead
 \\
Take a look at the histogram of \texttt{win}. Here we impose a
\textbf{normal curve} with the center being 0.5 and the spread, sd =
0.038. \\
\textbf{\texttt{ggplot} way:} \tiny \\
\texttt{r\ ggplot(data\_agg)\ +\ geom\_histogram(aes(x=win\_pct\_ave,\ y\ =\ ..density..),\ bins=10,\ fill=\ "blue"\ )\ +\ stat\_function(fun\ =\ dnorm,\ args\ =\ list(mean\ =\ mean(data\_agg\$win\_pct\_ave),\ sd\ =\ sd(data\_agg\$win\_pct\_ave)),\ colour\ =\ "red",\ \ \ \ \ \ \ \ \ \ \ \ \ \ \ \ \ \ \ \ \ \ \ \ \ \ \ \ \ \ \ \ \ \ \ \ \ \ \ \ \ \ \ \ size\ =\ 1.5)+\ labs(\ title\ =\ "Histogram\ of\ win\_pct\_ave",\ x\ =\ "win\_pct\_ave"\ ,\ y\ =\ "Frequency")} \\
\includegraphics{DataPreparationEDASlides_files/figure-latex/hist w normal-1.pdf} \\
\bottomrule
\end{longtable}

\frametitle{Normal variables}

The smoothed normal curve captures the shape of the histogram of win. Or
we will say that the variable \texttt{win} follows a normal distribution
approximately. Then we can describe the distribution of \texttt{win}
using the two numbers: mean and sd.

Roughly speaking:

\begin{itemize}
\item
  68\% of teams with \texttt{win} to be within one sd from the mean.
  \[ 0.5 \pm 0.038= [0.462, 
  0.538]\]
\item
  95\% of the teams with \texttt{win} to be within 2 sd from the mean:
  \[ 0.5 \pm 2 * 0.038= [0.425, 
  0.575]\]
\item
  2.5\% of the teams with \texttt{win} to be higher 2.5 times of sd
  above the mean: \[ > 0.5 + 2 * 0.038= 
  0.575\]
\end{itemize}

\begin{longtable}[]{@{}
  >{\raggedright\arraybackslash}p{(\columnwidth - 0\tabcolsep) * \real{0.06}}@{}}
\toprule
\endhead
 \\
\textbf{\texttt{ggplot} plots} (shown on next slide) \tiny \\
\texttt{r\ data\_agg\ \%\textgreater{}\%\ ggplot(aes(x\ =\ payroll\_total,\ y\ =\ win\_pct\_ave))\ +\ \#\ geometric\ options:\ color,\ size,\ shape,\ alpha:\ transparency\ (range:\ 0\ to\ 1)\ geom\_point(color\ =\ "blue",\ size=\ 3,\ alpha\ =\ .8)\ +\ geom\_text\_repel(aes(label\ =\ team),\ size\ =\ 3)\ +\ labs(title\ =\ "MLB\ Team\textquotesingle{}s\ Overall\ Win\ \ vs.\ Payroll",\ x\ =\ "Payroll\_total",\ y\ =\ "Win\_pct\_ave")} \\
\bottomrule
\end{longtable}

\frametitle{Explore variable relationships: Scatter plots  }

\textbf{We notice the positive association:}\\
\textbf{when \texttt{payroll\_total} increases, so does
\texttt{win\_pct\_ave}.}

\includegraphics{DataPreparationEDASlides_files/figure-latex/scatter p with team names 2-1.pdf}

\begin{longtable}[]{@{}
  >{\raggedright\arraybackslash}p{(\columnwidth - 0\tabcolsep) * \real{0.06}}@{}}
\toprule
\endhead
 \\
We can bring in other variables to adjust the color, size, and alpha of
the scatter plot via \textbf{aesthetic mapping}. \\
\includegraphics{DataPreparationEDASlides_files/figure-latex/scatter p with team names with mappings 2-1.pdf} \\
\bottomrule
\end{longtable}

\frametitle{Explore variable relationships}

\textbf{Least Squared Lines}

\begin{itemize}
\tightlist
\item
  The simplest function to capture the relationship between pay and
  performance is through the linear model.\\
\item
  We impose the least squared equation on top of the scatter plot using
  \texttt{ggplot()} with \texttt{geom\_smooth()}.\\
\item
  We also annotate the two teams \texttt{Oakland\ Athletics} and
  \texttt{Boston\ Red\ Sox}.
\end{itemize}

\begin{longtable}[]{@{}
  >{\raggedright\arraybackslash}p{(\columnwidth - 0\tabcolsep) * \real{0.06}}@{}}
\toprule
\endhead
 \\
\emph{Answer to Question 1:} \\
HERE is how the article concludes that Beane is worth as much as the GM
in Red Sox. By looking at the above plot, Oakland A's win pct is more or
less the same as that of Red Sox, so based on the LS equation, the team
should have paid 2 billion! \\
Do you agree with this argument? Why or why not? \\
\emph{Answer to Question 2:} \\
From this regression line, we see a clear upward trend. Or precisely the
least squared equation has a positive coefficient. Consequently, the
more a team is paid the better performance we expect the team has. \\
\bottomrule
\end{longtable}

\frametitle{Conclusions/Discussions}

\textbf{Questions for you:}

\begin{itemize}
\tightlist
\item
  Do you agree with the conclusions made based on a regression analysis
  shown above?\\
\item
  How would you carry out a study which may have done a better job? In
  what way?
\end{itemize}

\hypertarget{part-ii-analyze-pay-and-winning-percent-over-time-and-by-team}{%
\subsection{Part II: Analyze pay and winning percent over time and by
team}\label{part-ii-analyze-pay-and-winning-percent-over-time-and-by-team}}

\begin{itemize}
\tightlist
\item
  Payroll and performance varies depending on teams and years.\\
\item
  We investigate changes over time and by teams to see how payroll
  relates to performance.
\end{itemize}

\begin{longtable}[]{@{}
  >{\raggedright\arraybackslash}p{(\columnwidth - 0\tabcolsep) * \real{0.06}}@{}}
\toprule
\endhead
 \\
Summary statistics can not describe the distributions of either payroll
or performances. Back to back boxplots of payroll or winning percentage
would capture the variability in details. \\
\tiny \\
\texttt{r\ baseball\ \%\textgreater{}\%\ ggplot(aes(x\ =\ team,\ y\ =\ win\_pct,\ fill\ =\ team))\ +\ geom\_boxplot()\ +\ xlab("Team")\ +\ ylab("Winning\ percentage")\ +\ ggtitle("Winning\ percentage\ by\ team")\ +\ theme\_bw()\ +\ theme(legend.position\ =\ "none",\ \#\ adjust\ for\ margins\ around\ the\ plot;\ t:\ top;\ r:\ right;\ b:\ bottom;\ l:\ left\ plot.margin\ =\ margin(t\ =\ 5,\ r\ =\ 50,\ b\ =\ 5,\ l\ =\ 0,\ unit\ =\ "pt"),\ axis.text.x\ =\ element\_text(angle\ =\ -60,\ vjust\ =\ 0,\ hjust\ =\ 0))} \\
\bottomrule
\end{longtable}

\frametitle{Compare payroll and performance}

Summary statistics can not describe the distributions of either payroll
or performances. Back to back boxplots of payroll or winning percentage
would capture the variability in details.
\includegraphics{DataPreparationEDASlides_files/figure-latex/back to back boxplots 2-1.pdf}

\normalsize

We see clearly that the medians/means and spreads are very different. Is
there a more informative way to display this?

\begin{longtable}[]{@{}
  >{\raggedright\arraybackslash}p{(\columnwidth - 0\tabcolsep) * \real{0.06}}@{}}
\toprule
\endhead
 \\
We probably want to display the comparison by ranking the median for
example:
\includegraphics{DataPreparationEDASlides_files/figure-latex/unnamed-chunk-13-1.pdf} \\
\normalsize We see that \texttt{NY\ Yankees} and \texttt{Red\ Sox} are
consistently good teams while \texttt{Oakland\ A\textquotesingle{}s} has
a good overall team performance but the performance varies. \\
\bottomrule
\end{longtable}

\frametitle{Compare payroll and performance}

\begin{itemize}
\tightlist
\item
  Next: compare both \texttt{payroll} and \texttt{win\_pct} by teams.\\
\item
  Let us try to line up two back to back boxplots together.\\
\item
  Notice that we tried to rank one variable while carrying the other
  variable in the same order.\\
\item
  The hope is to reveal the relationship between \texttt{payroll} and
  \texttt{performance}.
\end{itemize}

\begin{longtable}[]{@{}
  >{\raggedright\arraybackslash}p{(\columnwidth - 0\tabcolsep) * \real{0.06}}@{}}
\toprule
\endhead
 \\
\includegraphics{DataPreparationEDASlides_files/figure-latex/unnamed-chunk-14-1.pdf} \\
Bingo! While \texttt{Oakland\ A\textquotesingle{}s} payroll are
consistently lower than that of \texttt{Red\ Sox}, they have similar
performance!!! \\
\bottomrule
\end{longtable}

\frametitle{Compare payroll and performance  }

Alternative boxplot faceting \tiny

\begin{Shaded}
\begin{Highlighting}[]
\CommentTok{\# use reorder\_within() and scale\_x\_reordered() from tidytext to order boxplot within each facet}
\FunctionTok{library}\NormalTok{(tidytext)}
\CommentTok{\# facet names}
\NormalTok{facet\_names }\OtherTok{\textless{}{-}} \FunctionTok{c}\NormalTok{(}\StringTok{"payroll"} \OtherTok{=} \StringTok{"Payroll"}\NormalTok{,}
                 \StringTok{"win\_pct"} \OtherTok{=} \StringTok{"Winning percentage"}\NormalTok{)}
\NormalTok{baseball }\SpecialCharTok{\%\textgreater{}\%}
  \FunctionTok{select}\NormalTok{(}\SpecialCharTok{{-}}\NormalTok{win\_num) }\SpecialCharTok{\%\textgreater{}\%}
  \FunctionTok{pivot\_longer}\NormalTok{(}\AttributeTok{cols =} \FunctionTok{c}\NormalTok{(}\StringTok{"payroll"}\NormalTok{, }\StringTok{"win\_pct"}\NormalTok{),}
               \AttributeTok{names\_to =} \StringTok{"variable"}\NormalTok{) }\SpecialCharTok{\%\textgreater{}\%}
  \FunctionTok{ggplot}\NormalTok{(}\FunctionTok{aes}\NormalTok{(}\AttributeTok{x =} \FunctionTok{reorder\_within}\NormalTok{(team, }\SpecialCharTok{{-}}\NormalTok{value, variable, }\AttributeTok{fun =}\NormalTok{ median), }
             \AttributeTok{y =}\NormalTok{ value, }\AttributeTok{fill =}\NormalTok{ team)) }\SpecialCharTok{+} 
  \FunctionTok{geom\_boxplot}\NormalTok{() }\SpecialCharTok{+}
  \FunctionTok{scale\_x\_reordered}\NormalTok{() }\SpecialCharTok{+}
  \FunctionTok{facet\_wrap}\NormalTok{(}\SpecialCharTok{\textasciitilde{}}\NormalTok{ variable, }\AttributeTok{ncol =} \DecValTok{1}\NormalTok{, }\AttributeTok{scales =} \StringTok{"free"}\NormalTok{,}
             \AttributeTok{labeller =} \FunctionTok{as\_labeller}\NormalTok{(facet\_names)) }\SpecialCharTok{+}
  \FunctionTok{xlab}\NormalTok{(}\StringTok{"Team"}\NormalTok{) }\SpecialCharTok{+} \FunctionTok{ylab}\NormalTok{(}\StringTok{""}\NormalTok{) }\SpecialCharTok{+}
  \FunctionTok{ggtitle}\NormalTok{(}\StringTok{"Payroll and winning percentage by team"}\NormalTok{) }\SpecialCharTok{+} 
\NormalTok{  boxplot\_theme}
\end{Highlighting}
\end{Shaded}

\begin{longtable}[]{@{}
  >{\raggedright\arraybackslash}p{(\columnwidth - 0\tabcolsep) * \real{0.06}}@{}}
\toprule
\endhead
 \\
A time series of performance may reveal patterns of performance over the
years to see if some teams are consistently better or worse. \tiny \\
\texttt{r\ payroll\_plot\ \textless{}-\ baseball\ \%\textgreater{}\%\ ggplot(aes(x\ =\ year,\ y\ =\ payroll,\ group\ =\ team,\ col\ =\ team))\ +\ geom\_line()\ +\ geom\_point()\ +\ theme\_bw()\ +\ ggtitle("Payroll\ versus\ year")\ payroll\_plot} \\
\includegraphics{DataPreparationEDASlides_files/figure-latex/unnamed-chunk-15-1.pdf} \\
\bottomrule
\end{longtable}

\frametitle{Comparing performance as a function of time  }

A time series of performance may reveal patterns of performance over the
years to see if some teams are consistently better or worse. \tiny

\begin{Shaded}
\begin{Highlighting}[]
\NormalTok{win\_pct\_plot }\OtherTok{\textless{}{-}}\NormalTok{ baseball }\SpecialCharTok{\%\textgreater{}\%} 
  \FunctionTok{ggplot}\NormalTok{(}\FunctionTok{aes}\NormalTok{(}\AttributeTok{x =}\NormalTok{ year, }\AttributeTok{y =}\NormalTok{ win\_pct, }\AttributeTok{group =}\NormalTok{ team, }\AttributeTok{col =}\NormalTok{ team)) }\SpecialCharTok{+} 
  \FunctionTok{geom\_line}\NormalTok{() }\SpecialCharTok{+} 
  \FunctionTok{geom\_point}\NormalTok{() }\SpecialCharTok{+}
  \FunctionTok{theme\_bw}\NormalTok{() }\SpecialCharTok{+}
  \FunctionTok{ggtitle}\NormalTok{(}\StringTok{"Winning percent versus year"}\NormalTok{)}
\NormalTok{win\_pct\_plot}
\end{Highlighting}
\end{Shaded}

\includegraphics{DataPreparationEDASlides_files/figure-latex/unnamed-chunk-16-1.pdf}

\begin{longtable}[]{@{}
  >{\raggedright\arraybackslash}p{(\columnwidth - 0\tabcolsep) * \real{0.06}}@{}}
\toprule
\endhead
 \\
Winning pct plot with only \texttt{NY\ Yankees} (blue),
\texttt{Boston\ Red\ Sox} (red) and \texttt{Oakland\ Athletics} (green)
while keeping all other teams as background in gray.
\includegraphics{DataPreparationEDASlides_files/figure-latex/unnamed-chunk-18-1.pdf} \\
\normalsize Now we see that \texttt{Red\ Sox} seems to perform better
most of the time compared to the
\texttt{Oakland\ A\textquotesingle{}s}. \\
\bottomrule
\end{longtable}

\frametitle{Performance, Payroll and Year}

We are trying to reveal the relationship between \texttt{performance}
and \texttt{payroll}. But it depends on which team at a given year.
\tiny

\begin{Shaded}
\begin{Highlighting}[]
\NormalTok{baseball }\SpecialCharTok{\%\textgreater{}\%}
  \FunctionTok{ggplot}\NormalTok{(}\FunctionTok{aes}\NormalTok{(}\AttributeTok{x=}\NormalTok{payroll, }\AttributeTok{y=}\NormalTok{win\_pct, }\AttributeTok{group =}\NormalTok{ team, }\AttributeTok{color=}\NormalTok{team)) }\SpecialCharTok{+}
  \FunctionTok{geom\_point}\NormalTok{()}\SpecialCharTok{+}
  \FunctionTok{geom\_smooth}\NormalTok{(}\AttributeTok{method=}\StringTok{"lm"}\NormalTok{, }\AttributeTok{formula=}\NormalTok{y}\SpecialCharTok{\textasciitilde{}}\NormalTok{x, }\AttributeTok{se=}\NormalTok{F,}\AttributeTok{color =} \StringTok{"red"}\NormalTok{)}\SpecialCharTok{+}
  \FunctionTok{facet\_wrap}\NormalTok{(}\SpecialCharTok{\textasciitilde{}}\NormalTok{team) }\SpecialCharTok{+} 
  \FunctionTok{theme\_bw}\NormalTok{() }\SpecialCharTok{+}
  \FunctionTok{theme}\NormalTok{(}\AttributeTok{legend.position =} \StringTok{"none"}\NormalTok{) }\SpecialCharTok{+}
  \FunctionTok{ggtitle}\NormalTok{(}\StringTok{"\textasciigrave{}payroll\textasciigrave{} vs \textasciigrave{}win\_pct\textasciigrave{} by team"}\NormalTok{)}
\end{Highlighting}
\end{Shaded}

\begin{longtable}[]{@{}
  >{\raggedright\arraybackslash}p{(\columnwidth - 0\tabcolsep) * \real{0.06}}@{}}
\toprule
\endhead
 \\
If we zoom in on a few teams we see a clear negative correlation between
payroll and performance. What is missing here? \tiny \\
\texttt{r\ baseball\ \%\textgreater{}\%\ filter(team\ \%in\%\ c("New\ York\ Yankees",\ "Boston\ Red\ Sox",\ "Oakland\ Athletics"))\ \%\textgreater{}\%\ ggplot(aes(x=payroll,\ y=win\_pct,\ group\ =\ team,\ color=team))\ +\ geom\_point()+\ geom\_smooth(method="lm",\ formula=\ y\textasciitilde{}x,\ \ se=F,color\ =\ "red")+\ facet\_wrap(\textasciitilde{}team)\ +\ theme\_bw()\ +\ theme(legend.position\ =\ "bottom")} \\
\includegraphics{DataPreparationEDASlides_files/figure-latex/unnamed-chunk-21-1.pdf} \\
\bottomrule
\end{longtable}

\frametitle{Performance, Payroll and Year}

We have seen before, \texttt{payroll} increases over years. It will be
better to examine \texttt{payroll} v.s. \texttt{win\_pct} by
\texttt{year}: \tiny

\begin{Shaded}
\begin{Highlighting}[]
\NormalTok{baseball }\SpecialCharTok{\%\textgreater{}\%}
  \FunctionTok{ggplot}\NormalTok{(}\FunctionTok{aes}\NormalTok{(}\AttributeTok{x=}\NormalTok{payroll, }\AttributeTok{y=}\NormalTok{win\_pct, }\AttributeTok{group =}\NormalTok{ year, }\AttributeTok{color=}\NormalTok{team)) }\SpecialCharTok{+}
  \FunctionTok{geom\_point}\NormalTok{()}\SpecialCharTok{+}
  \FunctionTok{geom\_smooth}\NormalTok{(}\AttributeTok{method=}\StringTok{"lm"}\NormalTok{, }\AttributeTok{formula=}\NormalTok{y}\SpecialCharTok{\textasciitilde{}}\NormalTok{x, }\AttributeTok{se=}\NormalTok{F,}\AttributeTok{color =} \StringTok{"red"}\NormalTok{)}\SpecialCharTok{+}
  \FunctionTok{facet\_wrap}\NormalTok{(}\SpecialCharTok{\textasciitilde{}}\NormalTok{year) }\SpecialCharTok{+} 
  \FunctionTok{theme\_bw}\NormalTok{() }\SpecialCharTok{+}
  \FunctionTok{theme}\NormalTok{(}\AttributeTok{legend.position =} \DecValTok{0}\NormalTok{)}
\end{Highlighting}
\end{Shaded}

\begin{longtable}[]{@{}
  >{\raggedright\arraybackslash}p{(\columnwidth - 0\tabcolsep) * \real{0.06}}@{}}
\toprule
\endhead
 \\
We can summarize the above three dimension plots via a movie that tracks
dynamic changes! \\
\alert{See the plotly movie in the html file!} \tiny \\
```r selected\_team \textless- c(``Oakland Athletics'', ``New York
Yankees'', ``Boston Red Sox'') \\
p \textless- baseball \%\textgreater\% ggplot(aes(x=payroll, y=win\_pct,
color=team, frame = year)) + theme(legend.position = 0) + geom\_point()
+ geom\_smooth(method=``lm'', formula=y\textasciitilde x, se=F, color =
``red'') + geom\_text(data = subset(baseball, team \%in\%
selected\_team), aes(label = team), show.legend = FALSE) +
theme\_bw() \\
ggplotly(p)
``\texttt{\textbackslash{}normalsize\ Perhaps\ we\ do\ not\ see\ strong\ evidence\ that}Oakland
A's\texttt{is\ comparable\ to}Red Sox` in performance. \\
\# Conclusions and Discussion - We have shown the power of exploratory
data analysis to reveal correlation between payroll and performance. -
Is payroll an important factor affecting the team performance if taking
more factors into account? - Here we assembled a dataset containing only
performance and payroll at the team level over a span of 17 yrs. -
Analysis via aggregated statistics can be misleading. - Substantial
variation can also exist within each team. - For example, payroll
distribution is drastically different. - See this article on MLB income
inequality:
\url{https://fivethirtyeight.com/features/good-mlb-teams-oppose-income-inequality/} \\
\bottomrule
\end{longtable}

\frametitle{Conclusions and Discussion}

Questions remain:

\begin{enumerate}
\def\labelenumi{\arabic{enumi}.}
\tightlist
\item
  Based on our current data,

  \begin{enumerate}
  \def\labelenumii{\alph{enumii})}
  \tightlist
  \item
    what model will you consider to capture effects of payroll, year and
    team over the performace?
  \item
    would you use other measurements as dependent variable, e.g.~annual
    payroll increase?
  \end{enumerate}
\item
  If you are asked to run the study to find out what are the main
  factors affecting performance, how would you do it?\\
  To narrow down the scope of the first step of the study, what
  information you may gather?
\end{enumerate}

\hypertarget{appendix-sample-statistics}{%
\section{Appendix: Sample Statistics}\label{appendix-sample-statistics}}

We remind readers of the definition of sample statistics here.

\begin{itemize}
\tightlist
\item
  Sample mean:
\end{itemize}

\[\bar{y} = \frac{1}{n}\sum_{i=1}^n y_i\]

\begin{itemize}
\tightlist
\item
  Sample variance:
\end{itemize}

\[s^2 = \frac{\sum_{i=1}^{n}(y_i - \bar{y})^2}{n-1}\]

\begin{center}\rule{0.5\linewidth}{0.5pt}\end{center}

\frametitle{Appendix: Sample Statistics}

\begin{itemize}
\tightlist
\item
  Sample Standard Deviation:
\end{itemize}

\[s = \sqrt\frac{\sum_{i=1}^{n}(y_i - \bar{y})^2} {n - 1}\]

\begin{itemize}
\tightlist
\item
  Sample correlation
\end{itemize}

\[r = \frac{\sum_{i=1}^n (x_i - \bar{x})(y_i - \bar{y})}{\sqrt{\sum_{i=1}^n (x_i - \bar{x})^2 \sum_{i=1}^n (y_i - \bar{y})^2}} = \frac{\sum_{i=1}^n (x_i - \bar{x})(y_i - \bar{y})}{s_x s_y} \]

\end{document}
